%%
%% This is file `emisa-author-template.tex',
%% generated with the docstrip utility.
%%
%% The original source files were:
%%
%% emisa.dtx  (with options: `template,article')
%% 
%% This is a template for a document to be used with emisa.cls.
%% Just copy (not move) this file to your work space, and rename and use
%% it as you see fit.
%% ------------------------------------------------------------------------
%% 
\documentclass[british]{emisa}
%% You can use this additional option (e.g.,"[english,draft]"):
%% draft -- this marks overfull lines
%% The following package imports are recommended, but not obligatory;
%% you might want take a look into their respective manuals if you
%% don't know what they do.
\usepackage{amsmath,amssymb,mathtools}
\usepackage{algorithmicx,algorithm}
%% Additional package imports go here:
%% Here, the normal text begins.
\begin{document}
\begin{article}{%
%% Please declare the title elements of your article here.  Unused
%% elements can either be deleted or commented out, or else just let
%% empty.  In either case they are not typeset.
%% If the option referee or review is given, all author tags, address,
%% email and acknowledgements will be likewise omitted.
  \title{}
  \subtitle{}
  \author*{<Name>}{<Email address>}
  \address{address line 1\\address line 2}
  \author{Name}
  \address[a]{}
  \abstract{}
  \keywords{Keyword 1 \and keyword 2\and keyword 3}
  \authornote{This article extends an earlier conference paper, see ...}
  \acknowledgements{}
%% Please declare here the bibliography data base(s) you want to use
%% in this article (make sure to add the file extension, e.g. .bib):
  \bibliography{}
  }
%% Please insert your article text here.
\section{Introduction}
\subsection{The research problem}
%% Remember to provide a unique label for each section, table, figure, listing and algorithm for referencing purposes.
%%
%% This directive typesets the bibliography.  To achieve this, one has
%% to run the biber program on the corresponding auxiliary file
%% generated in the previous LaTeX run; you can just use the job name
%% (the name of this file without ".tex")", e.g.: biber emisa-author-template
\printbibliography
\end{article}
\end{document}
\endinput
%%
%% End of file `emisa-author-template.tex'.
